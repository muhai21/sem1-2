\documentclass[12pt]{report}
\usepackage{times}
\usepackage{graphicx} % For including images
\graphicspath{{./}} % Set path to image directory (adjust if needed)
\title{Learning LaTeX}
\author{Muhaimin Kamran \\ Roll:366 \\ Mail: 20240659148muhaimin@juniv.edu}
\date{April 2025}

\begin{document}

\maketitle

\section*{Introduction}
LaTeX is a powerful typesetting system widely used for producing scientific and technical documents. This project aims to explore the fundamental aspects of writing a thesis using LaTeX on the Overleaf platform, highlighting its features and benefits for academic writing.

\section*{Summary of Today's Class}
\begin{itemize}
    \item Today, our professor conducted a surprise test.
    \item He emphasized the importance of being regular and punctual in class.
    \item The lecture focused on LaTeX, a typesetting system.
    \item We were introduced to Overleaf, an online LaTeX editor.
    \item The professor asked us to quickly write a LaTeX project.
\end{itemize}

\begin{figure}[h]
    \centering
    \includegraphics[width=0.5\textwidth]{overleaf.png} % Adjust width as needed
    \caption{Screenshot of the Overleaf interface used in class.}
    \label{fig:overleaf}
\end{figure}

\section*{Conclusion}
In conclusion, utilizing LaTeX through Overleaf streamlines the process of document preparation, offering tools that enhance collaboration and formatting. Mastering these skills is essential for producing high-quality academic work.

\section*{References}
\begin{enumerate}
    \item Overleaf. (n.d.). How to Write a Thesis in LaTeX (Part 1): Basic Structure. 
    \item Overleaf. (n.d.). Learn LaTeX in 30 minutes.
    \item Overleaf. (n.d.). Bibliography management in LaTeX. 
    \item Grok AI. (n.d.). AI-Powered Writing Assistance. 
    \item Blackbox AI. (n.d.). AI for Code Generation and Assistance.
\end{enumerate}



\end{document}
